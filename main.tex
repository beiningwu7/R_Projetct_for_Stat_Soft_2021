% Options for packages loaded elsewhere
\PassOptionsToPackage{unicode}{hyperref}
\PassOptionsToPackage{hyphens}{url}
%
\documentclass[
]{article}
\title{Statistical Software Project}
\author{Beining Wu}
\date{12/26/2021}

\usepackage{amsmath,amssymb}
\usepackage{lmodern}
\usepackage{iftex}
\ifPDFTeX
  \usepackage[T1]{fontenc}
  \usepackage[utf8]{inputenc}
  \usepackage{textcomp} % provide euro and other symbols
\else % if luatex or xetex
  \usepackage{unicode-math}
  \defaultfontfeatures{Scale=MatchLowercase}
  \defaultfontfeatures[\rmfamily]{Ligatures=TeX,Scale=1}
\fi
% Use upquote if available, for straight quotes in verbatim environments
\IfFileExists{upquote.sty}{\usepackage{upquote}}{}
\IfFileExists{microtype.sty}{% use microtype if available
  \usepackage[]{microtype}
  \UseMicrotypeSet[protrusion]{basicmath} % disable protrusion for tt fonts
}{}
\makeatletter
\@ifundefined{KOMAClassName}{% if non-KOMA class
  \IfFileExists{parskip.sty}{%
    \usepackage{parskip}
  }{% else
    \setlength{\parindent}{0pt}
    \setlength{\parskip}{6pt plus 2pt minus 1pt}}
}{% if KOMA class
  \KOMAoptions{parskip=half}}
\makeatother
\usepackage{xcolor}
\IfFileExists{xurl.sty}{\usepackage{xurl}}{} % add URL line breaks if available
\IfFileExists{bookmark.sty}{\usepackage{bookmark}}{\usepackage{hyperref}}
\hypersetup{
  pdftitle={Statistical Software Project},
  pdfauthor={Beining Wu},
  hidelinks,
  pdfcreator={LaTeX via pandoc}}
\urlstyle{same} % disable monospaced font for URLs
\usepackage[margin=1in]{geometry}
\usepackage{color}
\usepackage{fancyvrb}
\newcommand{\VerbBar}{|}
\newcommand{\VERB}{\Verb[commandchars=\\\{\}]}
\DefineVerbatimEnvironment{Highlighting}{Verbatim}{commandchars=\\\{\}}
% Add ',fontsize=\small' for more characters per line
\usepackage{framed}
\definecolor{shadecolor}{RGB}{248,248,248}
\newenvironment{Shaded}{\begin{snugshade}}{\end{snugshade}}
\newcommand{\AlertTok}[1]{\textcolor[rgb]{0.94,0.16,0.16}{#1}}
\newcommand{\AnnotationTok}[1]{\textcolor[rgb]{0.56,0.35,0.01}{\textbf{\textit{#1}}}}
\newcommand{\AttributeTok}[1]{\textcolor[rgb]{0.77,0.63,0.00}{#1}}
\newcommand{\BaseNTok}[1]{\textcolor[rgb]{0.00,0.00,0.81}{#1}}
\newcommand{\BuiltInTok}[1]{#1}
\newcommand{\CharTok}[1]{\textcolor[rgb]{0.31,0.60,0.02}{#1}}
\newcommand{\CommentTok}[1]{\textcolor[rgb]{0.56,0.35,0.01}{\textit{#1}}}
\newcommand{\CommentVarTok}[1]{\textcolor[rgb]{0.56,0.35,0.01}{\textbf{\textit{#1}}}}
\newcommand{\ConstantTok}[1]{\textcolor[rgb]{0.00,0.00,0.00}{#1}}
\newcommand{\ControlFlowTok}[1]{\textcolor[rgb]{0.13,0.29,0.53}{\textbf{#1}}}
\newcommand{\DataTypeTok}[1]{\textcolor[rgb]{0.13,0.29,0.53}{#1}}
\newcommand{\DecValTok}[1]{\textcolor[rgb]{0.00,0.00,0.81}{#1}}
\newcommand{\DocumentationTok}[1]{\textcolor[rgb]{0.56,0.35,0.01}{\textbf{\textit{#1}}}}
\newcommand{\ErrorTok}[1]{\textcolor[rgb]{0.64,0.00,0.00}{\textbf{#1}}}
\newcommand{\ExtensionTok}[1]{#1}
\newcommand{\FloatTok}[1]{\textcolor[rgb]{0.00,0.00,0.81}{#1}}
\newcommand{\FunctionTok}[1]{\textcolor[rgb]{0.00,0.00,0.00}{#1}}
\newcommand{\ImportTok}[1]{#1}
\newcommand{\InformationTok}[1]{\textcolor[rgb]{0.56,0.35,0.01}{\textbf{\textit{#1}}}}
\newcommand{\KeywordTok}[1]{\textcolor[rgb]{0.13,0.29,0.53}{\textbf{#1}}}
\newcommand{\NormalTok}[1]{#1}
\newcommand{\OperatorTok}[1]{\textcolor[rgb]{0.81,0.36,0.00}{\textbf{#1}}}
\newcommand{\OtherTok}[1]{\textcolor[rgb]{0.56,0.35,0.01}{#1}}
\newcommand{\PreprocessorTok}[1]{\textcolor[rgb]{0.56,0.35,0.01}{\textit{#1}}}
\newcommand{\RegionMarkerTok}[1]{#1}
\newcommand{\SpecialCharTok}[1]{\textcolor[rgb]{0.00,0.00,0.00}{#1}}
\newcommand{\SpecialStringTok}[1]{\textcolor[rgb]{0.31,0.60,0.02}{#1}}
\newcommand{\StringTok}[1]{\textcolor[rgb]{0.31,0.60,0.02}{#1}}
\newcommand{\VariableTok}[1]{\textcolor[rgb]{0.00,0.00,0.00}{#1}}
\newcommand{\VerbatimStringTok}[1]{\textcolor[rgb]{0.31,0.60,0.02}{#1}}
\newcommand{\WarningTok}[1]{\textcolor[rgb]{0.56,0.35,0.01}{\textbf{\textit{#1}}}}
\usepackage{graphicx}
\makeatletter
\def\maxwidth{\ifdim\Gin@nat@width>\linewidth\linewidth\else\Gin@nat@width\fi}
\def\maxheight{\ifdim\Gin@nat@height>\textheight\textheight\else\Gin@nat@height\fi}
\makeatother
% Scale images if necessary, so that they will not overflow the page
% margins by default, and it is still possible to overwrite the defaults
% using explicit options in \includegraphics[width, height, ...]{}
\setkeys{Gin}{width=\maxwidth,height=\maxheight,keepaspectratio}
% Set default figure placement to htbp
\makeatletter
\def\fps@figure{htbp}
\makeatother
\setlength{\emergencystretch}{3em} % prevent overfull lines
\providecommand{\tightlist}{%
  \setlength{\itemsep}{0pt}\setlength{\parskip}{0pt}}
\setcounter{secnumdepth}{-\maxdimen} % remove section numbering
\ifLuaTeX
  \usepackage{selnolig}  % disable illegal ligatures
\fi

\begin{document}
\maketitle

{
\setcounter{tocdepth}{2}
\tableofcontents
}
\hypertarget{preliminaries}{%
\section{Preliminaries}\label{preliminaries}}

\begin{Shaded}
\begin{Highlighting}[]
\NormalTok{data }\OtherTok{\textless{}{-}} \FunctionTok{read.csv}\NormalTok{(}\AttributeTok{file =} \StringTok{"data.csv"}\NormalTok{, }\AttributeTok{head=}\ConstantTok{TRUE}\NormalTok{, }\AttributeTok{fileEncoding =} \StringTok{"UTF8"}\NormalTok{)}
\FunctionTok{attach}\NormalTok{(data)}
\FunctionTok{library}\NormalTok{(mapchina)}
\FunctionTok{library}\NormalTok{(tidyverse)}
\end{Highlighting}
\end{Shaded}

\begin{verbatim}
## -- Attaching packages --------------------------------------- tidyverse 1.3.1 --
\end{verbatim}

\begin{verbatim}
## v ggplot2 3.3.5     v purrr   0.3.4
## v tibble  3.1.6     v dplyr   1.0.7
## v tidyr   1.1.4     v stringr 1.4.0
## v readr   2.1.1     v forcats 0.5.1
\end{verbatim}

\begin{verbatim}
## -- Conflicts ------------------------------------------ tidyverse_conflicts() --
## x dplyr::filter() masks stats::filter()
## x dplyr::lag()    masks stats::lag()
\end{verbatim}

\begin{Shaded}
\begin{Highlighting}[]
\FunctionTok{library}\NormalTok{(sf)}
\end{Highlighting}
\end{Shaded}

\begin{verbatim}
## Linking to GEOS 3.9.1, GDAL 3.2.3, PROJ 7.2.1; sf_use_s2() is TRUE
\end{verbatim}

\begin{Shaded}
\begin{Highlighting}[]
\FunctionTok{library}\NormalTok{(fitdistrplus)}
\end{Highlighting}
\end{Shaded}

\begin{verbatim}
## Loading required package: MASS
\end{verbatim}

\begin{verbatim}
## 
## Attaching package: 'MASS'
\end{verbatim}

\begin{verbatim}
## The following object is masked from 'package:dplyr':
## 
##     select
\end{verbatim}

\begin{verbatim}
## Loading required package: survival
\end{verbatim}

\hypertarget{data-overview}{%
\section{Data Overview}\label{data-overview}}

First we take a glance at the distrbution of the \texttt{rent} variable.
d`

\begin{Shaded}
\begin{Highlighting}[]
\FunctionTok{head}\NormalTok{(data)}
\end{Highlighting}
\end{Shaded}

\begin{verbatim}
##   rent bedroom livingroom bathroom area room floor_grp subway region  heating
## 1 2730       2          1        1   12 主卧    高楼层     是   通州 集中供暖
## 2 2740       3          1        1    9 次卧    低楼层     是   昌平 集中供暖
## 3 2810       3          1        1   14 主卧    低楼层     是   丰台 集中供暖
## 4 2650       4          1        1    8 次卧    低楼层     是   丰台 集中供暖
## 5 2670       4          1        1   13 主卧    高楼层     否   丰台 集中供暖
## 6 2530       3          1        1   12 次卧    高楼层     是   顺义 集中供暖
\end{verbatim}

\begin{Shaded}
\begin{Highlighting}[]
\FunctionTok{summary}\NormalTok{(data)}
\end{Highlighting}
\end{Shaded}

\begin{verbatim}
##       rent         bedroom        livingroom      bathroom          area      
##  Min.   :1150   Min.   :2.000   Min.   :1.00   Min.   :1.000   Min.   : 5.00  
##  1st Qu.:2240   1st Qu.:2.000   1st Qu.:1.00   1st Qu.:1.000   1st Qu.:10.00  
##  Median :2690   Median :3.000   Median :1.00   Median :1.000   Median :12.00  
##  Mean   :2798   Mean   :2.996   Mean   :1.01   Mean   :1.027   Mean   :12.85  
##  3rd Qu.:3230   3rd Qu.:4.000   3rd Qu.:1.00   3rd Qu.:1.000   3rd Qu.:15.00  
##  Max.   :6460   Max.   :5.000   Max.   :2.00   Max.   :2.000   Max.   :30.00  
##      room            floor_grp            subway             region         
##  Length:5149        Length:5149        Length:5149        Length:5149       
##  Class :character   Class :character   Class :character   Class :character  
##  Mode  :character   Mode  :character   Mode  :character   Mode  :character  
##                                                                             
##                                                                             
##                                                                             
##    heating         
##  Length:5149       
##  Class :character  
##  Mode  :character  
##                    
##                    
## 
\end{verbatim}

\begin{Shaded}
\begin{Highlighting}[]
\FunctionTok{hist}\NormalTok{(rent, }\AttributeTok{breaks =} \StringTok{"scott"}\NormalTok{, }\AttributeTok{main =} \StringTok{"Histogram of Rent"}\NormalTok{, }\AttributeTok{xlab =} \StringTok{"Monthly Rent Price"}\NormalTok{, }\AttributeTok{ylab =} \StringTok{"Frequency"}\NormalTok{, }\AttributeTok{freq =} \ConstantTok{FALSE}\NormalTok{)}
\FunctionTok{lines}\NormalTok{(}\FunctionTok{density}\NormalTok{(rent))}
\end{Highlighting}
\end{Shaded}

\includegraphics{main_files/figure-latex/unnamed-chunk-2-1.pdf}

\hypertarget{fitting}{%
\section{Fitting}\label{fitting}}

From the histogram plot we may hypothesis that the distribution is
gamma.

\begin{Shaded}
\begin{Highlighting}[]
\FunctionTok{library}\NormalTok{(fitdistrplus)}
\FunctionTok{library}\NormalTok{(MASS)}
\NormalTok{fit.gamma }\OtherTok{\textless{}{-}} \FunctionTok{fitdist}\NormalTok{(rent, }\AttributeTok{distr =} \StringTok{"gamma"}\NormalTok{, }\AttributeTok{method =} \StringTok{"mle"}\NormalTok{)}
\FunctionTok{summary}\NormalTok{(fit.gamma)}
\end{Highlighting}
\end{Shaded}

\begin{verbatim}
## Fitting of the distribution ' gamma ' by maximum likelihood 
## Parameters : 
##           estimate   Std. Error
## shape 14.212371899 0.1739956485
## rate   0.005080088 0.0000607289
## Loglikelihood:  -41215.48   AIC:  82434.97   BIC:  82448.06 
## Correlation matrix:
##           shape      rate
## shape 1.0000000 0.9549845
## rate  0.9549845 1.0000000
\end{verbatim}

\begin{Shaded}
\begin{Highlighting}[]
\FunctionTok{plot}\NormalTok{(fit.gamma)}
\end{Highlighting}
\end{Shaded}

\includegraphics{main_files/figure-latex/unnamed-chunk-3-1.pdf}

\hypertarget{mean-unit-value-in-each-district}{%
\section{Mean Unit Value in Each
District}\label{mean-unit-value-in-each-district}}

Now we consider the mean rent in each district.

\begin{Shaded}
\begin{Highlighting}[]
\NormalTok{unit }\OtherTok{\textless{}{-}}\NormalTok{ data}\SpecialCharTok{$}\NormalTok{rent}\SpecialCharTok{/}\NormalTok{data}\SpecialCharTok{$}\NormalTok{area}

\NormalTok{mean\_unit\_reg }\OtherTok{\textless{}{-}} \FunctionTok{c}\NormalTok{(}\FunctionTok{mean}\NormalTok{(unit[region}\SpecialCharTok{==}\StringTok{"东城"}\NormalTok{]),}\FunctionTok{mean}\NormalTok{(unit[region}\SpecialCharTok{==}\StringTok{"西城"}\NormalTok{]),}\FunctionTok{mean}\NormalTok{(unit[region}\SpecialCharTok{==}\StringTok{"昌平"}\NormalTok{]),}\FunctionTok{mean}\NormalTok{(unit[region}\SpecialCharTok{==}\StringTok{"大兴"}\NormalTok{]),}\FunctionTok{mean}\NormalTok{(unit[region}\SpecialCharTok{==}\StringTok{"房山"}\NormalTok{]),}\FunctionTok{mean}\NormalTok{(unit[region}\SpecialCharTok{==}\StringTok{"怀柔"}\NormalTok{]),}\FunctionTok{mean}\NormalTok{(unit[region}\SpecialCharTok{==}\StringTok{"门头沟"}\NormalTok{]),}\FunctionTok{mean}\NormalTok{(unit[region}\SpecialCharTok{==}\StringTok{"密云"}\NormalTok{]),}\FunctionTok{mean}\NormalTok{(unit[region}\SpecialCharTok{==}\StringTok{"平谷"}\NormalTok{]),}\FunctionTok{mean}\NormalTok{(unit[region}\SpecialCharTok{==}\StringTok{"顺义"}\NormalTok{]),}\FunctionTok{mean}\NormalTok{(unit[region}\SpecialCharTok{==}\StringTok{"通州"}\NormalTok{]),}\FunctionTok{mean}\NormalTok{(unit[region}\SpecialCharTok{==}\StringTok{"延庆"}\NormalTok{]),}\FunctionTok{mean}\NormalTok{(unit[region}\SpecialCharTok{==}\StringTok{"朝阳"}\NormalTok{]),}\FunctionTok{mean}\NormalTok{(unit[region}\SpecialCharTok{==}\StringTok{"丰台"}\NormalTok{]),}\FunctionTok{mean}\NormalTok{(unit[region}\SpecialCharTok{==}\StringTok{"海淀"}\NormalTok{]),}\FunctionTok{mean}\NormalTok{(unit[region}\SpecialCharTok{==}\StringTok{"石景山"}\NormalTok{]))}


\NormalTok{mean\_unit\_reg[}\FunctionTok{is.na}\NormalTok{(mean\_unit\_reg)]}\OtherTok{=}\DecValTok{0}
\end{Highlighting}
\end{Shaded}

And we can visualize as follows.

\begin{Shaded}
\begin{Highlighting}[]
\NormalTok{df }\OtherTok{\textless{}{-}}\NormalTok{ china }\SpecialCharTok{\%\textgreater{}\%}
        \FunctionTok{filter}\NormalTok{(Name\_Province }\SpecialCharTok{==} \StringTok{"北京市"}\NormalTok{)}

\FunctionTok{ggplot}\NormalTok{(}\AttributeTok{data =}\NormalTok{ df) }\SpecialCharTok{+}
        \FunctionTok{geom\_sf}\NormalTok{(}\FunctionTok{aes}\NormalTok{(}\AttributeTok{fill =} \FunctionTok{rank}\NormalTok{(mean\_unit\_reg))) }\SpecialCharTok{+}
        \FunctionTok{scale\_fill\_distiller}\NormalTok{(}\AttributeTok{palette =} \StringTok{"BuPu"}\NormalTok{, }\AttributeTok{direction =} \DecValTok{1}\NormalTok{) }\SpecialCharTok{+}
        \FunctionTok{theme\_bw}\NormalTok{() }\SpecialCharTok{+}
        \FunctionTok{theme}\NormalTok{(}\AttributeTok{legend.position =} \StringTok{"none"}\NormalTok{)}
\end{Highlighting}
\end{Shaded}

\includegraphics{main_files/figure-latex/unnamed-chunk-5-1.pdf}

From the map we know that the unit price is significantly
\textbf{higher} when location is more central. Now the

\hypertarget{mean-unit-value-and-subway}{%
\section{Mean Unit Value and Subway}\label{mean-unit-value-and-subway}}

\begin{Shaded}
\begin{Highlighting}[]
\FunctionTok{boxplot}\NormalTok{(rent}\SpecialCharTok{\textasciitilde{}}\NormalTok{subway, }\AttributeTok{main=}\StringTok{"Boxplot of Rent, Grouing by Subway"}\NormalTok{)}
\end{Highlighting}
\end{Shaded}

\begin{verbatim}
## Warning in axis(side = base::quote(1), at = base::quote(1:2), labels =
## base::quote(c("否", : conversion failure on '否' in 'mbcsToSbcs': dot
## substituted for <e5>
\end{verbatim}

\begin{verbatim}
## Warning in axis(side = base::quote(1), at = base::quote(1:2), labels =
## base::quote(c("否", : conversion failure on '否' in 'mbcsToSbcs': dot
## substituted for <90>
\end{verbatim}

\begin{verbatim}
## Warning in axis(side = base::quote(1), at = base::quote(1:2), labels =
## base::quote(c("否", : conversion failure on '否' in 'mbcsToSbcs': dot
## substituted for <a6>
\end{verbatim}

\begin{verbatim}
## Warning in axis(side = base::quote(1), at = base::quote(1:2), labels =
## base::quote(c("否", : conversion failure on '否' in 'mbcsToSbcs': dot
## substituted for <e5>
\end{verbatim}

\begin{verbatim}
## Warning in axis(side = base::quote(1), at = base::quote(1:2), labels =
## base::quote(c("否", : conversion failure on '否' in 'mbcsToSbcs': dot
## substituted for <90>
\end{verbatim}

\begin{verbatim}
## Warning in axis(side = base::quote(1), at = base::quote(1:2), labels =
## base::quote(c("否", : conversion failure on '否' in 'mbcsToSbcs': dot
## substituted for <a6>
\end{verbatim}

\begin{verbatim}
## Warning in axis(side = base::quote(1), at = base::quote(1:2), labels =
## base::quote(c("否", : conversion failure on '是' in 'mbcsToSbcs': dot
## substituted for <e6>
\end{verbatim}

\begin{verbatim}
## Warning in axis(side = base::quote(1), at = base::quote(1:2), labels =
## base::quote(c("否", : conversion failure on '是' in 'mbcsToSbcs': dot
## substituted for <98>
\end{verbatim}

\begin{verbatim}
## Warning in axis(side = base::quote(1), at = base::quote(1:2), labels =
## base::quote(c("否", : conversion failure on '是' in 'mbcsToSbcs': dot
## substituted for <af>
\end{verbatim}

\begin{verbatim}
## Warning in axis(side = base::quote(1), at = base::quote(1:2), labels =
## base::quote(c("否", : conversion failure on '是' in 'mbcsToSbcs': dot
## substituted for <e6>
\end{verbatim}

\begin{verbatim}
## Warning in axis(side = base::quote(1), at = base::quote(1:2), labels =
## base::quote(c("否", : conversion failure on '是' in 'mbcsToSbcs': dot
## substituted for <98>
\end{verbatim}

\begin{verbatim}
## Warning in axis(side = base::quote(1), at = base::quote(1:2), labels =
## base::quote(c("否", : conversion failure on '是' in 'mbcsToSbcs': dot
## substituted for <af>
\end{verbatim}

\includegraphics{main_files/figure-latex/unnamed-chunk-6-1.pdf}

\end{document}
